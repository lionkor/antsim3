\documentclass[12pt,a4paper]{article}

\usepackage{listings}
\usepackage{xcolor}
\usepackage[colorlinks=true]{hyperref}
\usepackage{makeidx}

\definecolor{color_keyword}{rgb}{0,0.6,0}
\definecolor{color_comment}{rgb}{0.5,0.5,0.5}
\definecolor{color_string}{rgb}{0.58,0,0.82}
\definecolor{color_number}{rgb}{0.58,0,0.82}
\definecolor{color_background}{rgb}{0.95,0.95,0.95}

\lstdefinestyle{mystyle}{
    backgroundcolor=\color{color_background},   
    commentstyle=\color{color_comment},
    keywordstyle=\color{color_keyword},
    numberstyle=\tiny\color{color_number},
    stringstyle=\color{color_string},
    basicstyle=\ttfamily\footnotesize,
    breakatwhitespace=false,
    breaklines=true,
    captionpos=b,
    keepspaces=true,
    numbers=left,
    numbersep=5pt,
    showspaces=false,
    showstringspaces=false,
    showtabs=false,                  
    tabsize=3
}
\lstset{style=mystyle}

\title{Engine Scripting with Lua}
\author{Lion Kortlepel \texttt{@lionkor}}

\begin{document}

\begin{titlepage}
\clearpage\maketitle
\thispagestyle{empty}
\end{titlepage}

\pagenumbering{roman}

\tableofcontents
\pagebreak

\begin{abstract}
This engine lets you script behavior in \href{https://www.lua.org/about.html}{Lua}. 
In order for this to be useful, the engine provides a number of functions and globals, which are documented in this PDF. Further, it will show how to use Lua to script some basic behavior by providing some examples.

\end{abstract}

\pagebreak

\pagenumbering{arabic}
\setcounter{page}{1}

\section{Setup}

For scripts to be read by the engine, an \texttt{Entity} needs to have a  \texttt{ScriptableComponent}. The constructor of the latter accepts a scriptfile name, like \texttt{example.lua}. For the file to be loaded it needs to be in the \texttt{Data/} directory and listed in the \texttt{Data/res.list} of your project.

An example program follows that will be used for the rest of this pdf.

\begin{lstlisting}[language=C++,title=example.cpp]
#include "Engine.h"
int main() {
	// Create an application
	Application app("Scripting101 Program", { 800, 600 });
	// Create an entity
	WeakPtr<Entity> entity_weak = app.add_entity();
	// Lock the Ptr for temporary access
	auto entity = entity_weak.lock();
	// Add ScriptableComponent
	entity->add_component<ScriptableComponent>("example.lua");
	// Run the application
	return app.run();
}
\end{lstlisting}

Additionally, the files \texttt{Data/example.lua} and \texttt{Data/res.list} exist.

\begin{lstlisting}[language={[5.0]Lua},title=Data/example.lua]
Engine.log_info("Hello, World!")
\end{lstlisting}

\begin{lstlisting}[title=Data/res.list]
example.lua
\end{lstlisting}

With these files in place and the \texttt{example.cpp} compiled, we can now write code in \texttt{Data/example.lua}.

\pagebreak

\section{Exposed Functions}

\subsection{Engine API}

The \texttt{Engine} namespace provides general engine functionality, mostly used for debugging and core engine functionalities.

\subsubsection{Engine.log\_info(message)}

\begin{lstlisting}[language={[5.0]Lua},title=Data/example.lua]
Engine.log_info("Hello, World!")
\end{lstlisting}


\end{document}